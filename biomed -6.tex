\documentclass[12pt]{article}

\usepackage{graphicx}
\graphicspath{{ASHISH}}

\begin{document}

\begin{figure}

\centering
\includegraphics[scale=1]{collage.png}

\end{figure}

\section*{ASSIGNMENT - 6 of BioMedical Engineering\\"5 Solution to Covid19 Provided by Biomedical Engineers"\\\\Submitted by :ASHISH\\Roll.no :21111013\\Details :1st Semester ,BioMedical Engineering\\Supervision by :SAURABH GUPTA SIR}

\clearpage
\section*{ 1 .INTRODUCTION:}
Engineers as capable and necessary responders to pandemics has been demonstrated throughout the ages. The design skills, systems approach, and innovative mindset that engineers bring all have the potential to combat crises in novel and impactful ways. And when a disparities lens is applied, a lens that views gaps in access, resources, and care, the engineering solutions are bound to be more robust and equitable. Equipping engineers to effectively apply a disparities lens requires them to understand the complexities and implications of racial disparities to include the public health concerns and the historical context of differential medical treatment based on race.

\section*{ 2 .Engineering Solution to Covid}
Posing the response to COVID-19 as a systems problem, engineers have effectively collaborated with scientists and clinicians to understand, limit spread, and contribute to the development of a novel vaccine and effective therapies. They also turned their attention to models and simulations to better understand transmission through aerosol droplet and patterns of transmission.

\section*{3. Engineering Solutions to Health Disparities }
A systems approach to health disparities by engineers affords a special opportunity to merge the development of innovative technologies with unmet health needs fueled by structural racism and social determinants of health. Without a lens through which inequities that shape disease processes and access to care, designs that are accessible and fully effective for all populations will be lacking.

Partnerships between engineering and medicine at programmatic levels constitute another mechanism to provide training and work for engineers in the health disparities arena. 

\section*{4.Call to Action }
Engineers are poised and ready to address COVID-19. Deploying that talent with an eye toward elimination of disparities can and will have lasting implications related to the COVID-19 contagion and its disproportionate toll on Blacks. Engineers must work hand-in-hand with physicians to expand access to testing, protective equipment, and therapies. At present, our engineering workforce lacks sufficient diversity. Efforts to recruit and retain a diverse cadre must be taken on with a renewed sense of urgency.

\section*{ 4 .Compliance with Ethical Standards }
Examining engineering solutions to COVID along with engineering solutions to health disparities affords an opportunity to operate at the intersection to realize better solutions. Many of the ways in which engineers have provided and continue to provide solutions to COVID-19 are described in the following section. In many cases, these solutions would be enhanced with the application of a health disparities lens to ensure that solutions are designed such that all populations are able to receive benefit. Equipping engineers to apply a health disparities lens requires education and practice. Examples are included in the section on engineering solutions to health disparities. 

\section*{5 . Conclution :}
The role of engineers in response to the COVID-19 pandemic and in the elimination of health disparities, while not always visible, has important implications for the attainment of impactful solutions. The design skills, systems approach, and innovative mindset that engineers bring all have the potential to combat crises in novel and impactful ways. When a disparities lens is applied, a lens that views gaps in access, resources, and care, the engineering solutions are bound to be more robust and equitable.

Engineers working collaboratively with physicians and healthcare providers are poised to close equity gaps and strengthen the collective response to COVID-19 and future pandemics.




\end{document}